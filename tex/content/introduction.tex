\section{Введение}
    Задача деревенского почтальона (\textsf{RPP}) является расширением более известной задачи о китайском почтальоне (\textsf{Chinese Postman Problem - CPP}), в которой необходимо найти цикл минимальной стоимости, проходящий через все ребра. В \textsf{RPP} же задается множество необходимых для посещения ребер. Задача имеет прикладное значение, например, для доставки почты, сборки мусора вдоль улиц, проверки на исправность трубопровода и для многих других. Эти задачи, как и задача о коммивояжере (\textsf{TSP}), являются частными случаями более общей задачи \textsf{Vehicle Routing Problem} [1]. 
    
    Впервые \textsf{RPP} была выдвинута C. S. Orloff [1], и сразу привлекла к себе интерес ученых.
    В общем случаем \textsf{RPP} является \textbf{NP}-полной, к ней сводится задача о коммивояжере.
    Однако при некоторм ограничени на множество ребер, которые необходимо посетить, существует полиномиальный алгоритм поиска решения. Это ограничение заключается в том, что количество компонент связности графа на множестве ребер, которое необходимо посетить является фиксированной константой $c > 0$, то есть этот граф не является связным. Тогда время работы алгоритма будет составлять  $O(n^{2c+1})$. 