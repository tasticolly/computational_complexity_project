\newpage
\section{Основная часть}
\subsection{Доказательство \textbf{NP}-полноты задачи о деревенском почтальоне (Утверждение 1)}
Поставленная задача может быть переформулирована в задачу принятия решения следующим образом: \\
    Пусть в дополнения к предыдущим входным данным дается некоторое положительное число $K$. Существует ли решение исходной задачи при условии, что стоимость цикла должна быть меньше $K$? 

\paragraph{Принадлежность к классу $\bf{NP}$.}
$\textsf{RPP} \in \textbf{NP}$, так как недетерминированная машина Тьюринга может за полиномиальное время проверить, существует ли проходящий по ребрам из $R$ цикл, сумма весов которого меньше $K$.

\paragraph{$\bf{NP}$-трудность.}
К задаче \textsf{RPP} сводится задача коммивояжёра (TCP из Определения 2). Пусть $(G,w)$ — пример задачи \textsf{TCP}, где $G$ — граф и $w$ - стоимости рёбер. Сведём её к \textsf{RPP} $(G', w', R)$:
граф $G'$ получается из $G$ добавлением петли нулевой стоимости к каждой вершине. Множество $R$ состоит из добавленных петель. Пусть $W$ - решение \textsf{TCP} $(G, w)$, тогда $W' = W \sqcup R$ - решение  для \textsf{RPP} $(G', w', R)$, так как путь будет содержать $R$, а петли не добавят стоимости, поэтому цикл будет минимальной стоимости. С другой стороны, если $W$ не является решением \textsf{TCP} $(G, w)$, то $W' = W \sqcup R$ - не будет являться решением  для \textsf{RPP} $(G', w', R)$. Таким образом, \textbf{NP}-полнота задачи о \textsf{деревенском почтальоне доказана.}

