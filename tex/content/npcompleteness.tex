\newpage
\section{Основная часть}
\subsection{Доказательство \textbf{NP}-полноты задачи о деревенском почтальоне (Утверждение 1)}
Поставленная задача может быть переформулирована в задачу принятия решения следующим образом: \\
    Пусть в дополнения к предыдущим входным данным дается некоторое положительное число $K$. Существует ли решение исходной задачи при условии, что стоимость цикла должна быть меньше $K$? 

\paragraph{Принадлежность к классу $\bf{NP}$.}
$\textsf{RPP} \in \textbf{NP}$, так как недетерминированная машина Тьюринга может за полиномиальное время проверить, существует ли проходящий по ребрам из $R$ цикл, сумма весов которого меньше $K$.

\paragraph{$\bf{NP}$-трудность.}
К задаче \textsf{RPP} сводится задача коммивояжёра (TCP из Определения 2). Пусть $(G,w)$ — пример задачи \textsf{TCP}, где $G$ — граф и $w$ - стоимости рёбер. Сведём её к \textsf{RPP} $(G', w', R)$:
граф $G'$ получается из $G$ добавлением петли нулевой стоимости к каждой вершине. Множество $R$ состоит из добавленных петель. Пусть $W$ - решение \textsf{TCP} $(G, w)$, тогда $W' = W \sqcup R$ - решение  для \textsf{RPP} $(G', w', R)$, так как путь будет содержать $R$, а петли не добавят стоимости, поэтому цикл будет минимальной стоимости. С другой стороны, если $W$ не является решением \textsf{TCP} $(G, w)$, то $W' = W \sqcup R$ - не будет являться решением  для \textsf{RPP} $(G', w', R)$. Таким образом, \textbf{NP}-полнота задачи о \textsf{деревенском почтальоне доказана.}

\subsection{Алгоритм решения задачи и его анализ}

Заметим, что \textsf{RPP}(G', w', R) сводится к (G, w, R), где G - полный граф, w - метрическая функция весов, то есть для любых трёх вершин $x$, $y$ и $z$ выполнено $w(x, z) \leq w(x, y) + w(y, z))$. \\Для этого: \\
 \par1) Добавим ко всем ребрам графа $G'$ максимальный вес ребра в графе $w_{max}' = max_{x,y} w'(x, y)$. Тогда неравенство треугольника будет выполнено: $w(x, y) = w'(x, y) + w_{max}' \leq 2w_{max}' \leq 2w_{max}' + w'(x, z) + w'(z, y) = w(x, z) + w(z, y)$. Цикл минимальной стоимости при этом останется тем же. Предположим, решение $W$ не совпадает с $W'$, так как ко всем ребрам прибавилось одинаковое значение, это значит, что длина цикла $W'$ отличается от длины цикла $W$. Пусть $ m = |W'|, n =|W|$, $cost(A)$ - суммарная стоимость ребер из множества $A$ с функцией весов $w$, а $cost'(A)$ - с функцией весов $w'$. Предположим $m < n$, тогда 
$cost'(W) = cost(W) - n \dotproduct w_{max}' < cost(W') - m \dotproduct w_{max}' = cost'(W')$, противоречие с тем, что W' - решение исходной задачи. В случае $m > n$ аналогично получим противоречие.\\
\par2) Для пар вершин, между которыми ребро отсутствует, проводим ребро со стоимостью $\infty$. Тогда они не могут содержаться в решении, так как цикл не будут минимальным.

\begin{lemma}
Пусть $(G = (V, E), w, R)$ - задача \textsf{RPP}, $c > 0$ - число компенент связности в $G\langle R \rangle$, а $W$ - решение задачи. Тогда $W=R \sqcup T \sqcup M$, где  
$T$ - множество из $ c - 1$ ребра такое, что  $G \left\langle R \sqcup T \right\rangle$ - связный, 
$M$ - совершенное паросочетание на вершинах нечетной степени в $G \left\langle R \sqcup T \right\rangle$.

\end{lemma}
\paragraph{Доказательство.}
Разобьем $W \setminus R$ на $T \sqcup M$ так, что $T$ - минимальное по включению множество рёбер, что $G \left\langle R \sqcup T \right\rangle$ - связный. Заметим, что размер множества равен $c - 1$, так как достаточно упорядочить компоненты связности и по очереди соединить.  
Докажем, что $M$ - паросочетание. Пусть нет, тогда найдем два ребра $(u,v),(v,w) \in M$ и заменим их на одно ребро $(u, w)$, ответ не станет хуже по неравенству треугольника. Так как в $W$ нет вершин нечётной степени(так как это цикл), то $M$ - паросочетание на вершинах нечётных степеней из $G \left\langle R \sqcup T \right\rangle$.

\paragraph{Доказательство Теоремы 1.}
Будем перебирать всевозможные множества $T$ из леммы. Таких не более чем $n^{2c−2}$, так как для каждого из $c-1$ ребра выбираем по 2 вершины из соответствующих компонент связности, размер которых не превосходит n. Для каждого множества $T$ вычисляем совершенное паросочетание минимального веса $M$ в графе на нечётных вершинах в $G \left\langle R \sqcup T \right\rangle$ за время $O(n^3)$. Из всех перебранных $T$ и $M$ возвращаем решение $R \sqcup M \sqcup T$ минимальной стоимости.