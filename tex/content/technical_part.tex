\newpage
\section{Техническая часть}

\subsection{Постановка задачи}
\begin{definition}
\textbf{Задача о деревенском почтальоне (\textsf{RPP})} \\
Дан граф $G=(V,E)$, веса рёбер $w:E \rightarrow N$, и множество рёбер $R \subset E$. Требуется найти цикл минимального суммарного веса, хотя бы один раз проходящий через каждое ребро из $R$.
\end{definition}

\begin{statement}
Поставленная задача $\bf{NP}$-полна.
\end{statement}

\begin{theorem}
 Дана задача $RPP(G, w, R)$. Пусть $G\langle R \rangle$ - граф, построенный на ребрах $R$, $c > 0$ - число компонент связности в $G \langle R \rangle$. Тогда задача о сельском почтальоне решается за $O(n^{2c-2} \cdot n^3)$. 
\end{theorem}

\subsection{Вспомогательные определения и утверждения}

\textbf{Замечание к определению 1.} \\
В случае, если множество необходимых для посещения ребер $R$ совпадает со всеми ребрами графа $G$, задача называется \textit{задачей о китайском почтальоне (\textsf{CPP})}.

\begin{definition}
\textbf{Задача о коммивояжере (\textsf{TCP})} \\
Дан граф $G = (V, E)$ и веса
на рёбрах $w : E \rightarrow R_+$ . Требуется найти гамильтонов цикл минимального веса.
\end{definition}

\textbf{Замечание к определению 2.} \\
В случае, если граф $G$ полный, а функция весов метрическая (то есть для
любых трёх вершин $x$, $y$ и $z$ выполнено $w(x, z) \leq w(x, y) + w(y, z))$, то задача называется \textit{метрической задачей о коммивояжере}.
