\subsection{Алгоритм решения задачи и его анализ}

Заметим, что \textsf{RPP}(G', w', R) сводится к (G, w, R), где G - полный граф, w - метрическая функция весов, то есть для любых трёх вершин $x$, $y$ и $z$ выполнено $w(x, z) \leq w(x, y) + w(y, z))$. \\Для этого: \\
 \par1) Добавим ко всем ребрам графа $G'$ максимальный вес ребра в графе $w_{max}' = max_{x,y} w'(x, y)$. Тогда неравенство треугольника будет выполнено: $w(x, y) = w'(x, y) + w_{max}' \leq 2w_{max}' \leq 2w_{max}' + w'(x, z) + w'(z, y) = w(x, z) + w(z, y)$. Цикл минимальной стоимости при этом останется тем же. Предположим, решение $W$ не совпадает с $W'$, так как ко всем ребрам прибавилось одинаковое значение, это значит, что длина цикла $W'$ отличается от длины цикла $W$. Пусть $ m = |W'|, n =|W|$, $cost(A)$ - суммарная стоимость ребер из множества $A$ с функцией весов $w$, а $cost'(A)$ - с функцией весов $w'$. Предположим $m < n$, тогда 
$cost'(W) = cost(W) - n \dotproduct w_{max}' < cost(W') - m \dotproduct w_{max}' = cost'(W')$, противоречие с тем, что W' - решение исходной задачи. В случае $m > n$ аналогично получим противоречие.\\
\par2) Для пар вершин, между которыми ребро отсутствует, проводим ребро со стоимостью $\infty$. Тогда они не могут содержаться в решении, так как цикл не будут минимальным.

\begin{lemma}
Пусть $(G = (V, E), w, R)$ - задача \textsf{RPP}, $c > 1$ - число компенент связности в $G\langle R \rangle$, а $W$ - решение задачи. Тогда $W=R \sqcup T \sqcup M$, где  
$T$ - множество из $ c - 1$ ребра такое, что  $G \left\langle R \sqcup T \right\rangle$ - связный, 
$M$ - совершенное паросочетание на вершинах нечетной степени в $G \left\langle R \sqcup T \right\rangle$.

\end{lemma}
\paragraph{Доказательство.}
Разобьем $W \setminus R$ на $T \sqcup M$ так, что $T$ - минимальное по включению множество рёбер, что $G \left\langle R \sqcup T \right\rangle$ - связный. Заметим, что размер множества равен $c - 1$, так как достаточно упорядочить компоненты связности и по очереди соединить.  
Докажем, что $M$ - паросочетание. Пусть нет, тогда найдем два ребра $(u,v),(v,w) \in M$ и заменим их на одно ребро $(u, w)$, ответ не станет хуже по неравенству треугольника. Так как в $W$ нет вершин нечётной степени(так как это цикл), то $M$ - паросочетание на вершинах нечётных степеней из $G \left\langle R \sqcup T \right\rangle$.

\paragraph{Доказательство Теоремы 1.}

Будем перебирать всевозможные множества $T$ из леммы. Таких не более чем $n^{2c-2}$, так как для каждого из $c-1$ ребра выбираем по 2 вершины из соответствующих компонент связности, размер которых не превосходит n. Для каждого множества $T$ вычисляем совершенное паросочетание минимального веса $M$ в графе на нечётных вершинах в $G \left\langle R \sqcup T \right\rangle$ за время $O(n^3)$. Из всех перебранных $T$ и $M$ возвращаем решение $R \sqcup M \sqcup T$ минимальной стоимости.



\begin{algorithm}
\caption{Основная функция(Псевдокод)}\label{alg:cap}
\begin{lstlisting}[language=Python]

def main(graph, requiredSubgraph):
    #Floyd algorithm to find 
    #distance between every pair
    vector<vector<int>> distances = graph.Floyd()
    #degrees of vertices in a graph G<R> 
    #constructed on edges from R.
    vector<int> degrees = requiredSubgraph.GetDegrees(); 
    int minimalCycleWeight = INF
    vector<pair<int, int>> res(
        requiredSubgraph.GetComponentsReference().size()
    );
    # index of connected component 
    const int firstComponentIndex = 1; 

    # trying to find best set T 
    # and perfect matching M by brute force
    chooseVariant(
        requiredSubgraph.GetComponents(),#get connective components  
        dist, 
        firstComponentIndex, 
        res, 
        degrees, 
        minimalCycleWeight
    );

    for ([u, v]: requiredSubgraph) {
        minimalCycleWeight += dist[u][v];
    }
    return minimalCycleWeight
\end{lstlisting}

\end{algorithm}


\begin{algorithm}
\caption{Функция перебора вариантов(Псевдокод)}\label{alg:cap}
\begin{lstlisting}[language=Python, basicstyle=\ttfamily\small)]
def chooseVariant(
    components
    dist, 
    firstComponentIndex, 
    res, 
    degrees, 
    minimalCycleWeight
):
    if (firstComponentIndex == components.size()) {
        int cur = 0;
        for ([u, v]: res)
            cur += gr[u][v];
        #Kun's algorythm for min cost matching
        cur += kun(gr, degrees); 
        ans = std::min(ans, cur);
        return;
    }
    for (int idx1 = 0; 
        idx1 < components[firstComponentIndex].size(); ++idx1) {
        for (int secondComponentIndex = 0; secondComponentIndex 
            < firstComponentIndex; ++secondComponentIndex) {
            for (int idx2 = 0; idx2 
                < components[compopnentIdx2].size(); ++idx2) {

                int u = components[firstComponentIndex][idx1];
                int v = components[secondComponentIndex][idx2];

                res[firstComponentIndex] = {u, v};

                ++degrees[u];
                ++degrees[v];

                chooseVariant(
                    components,
                    gr,
                    firstComponentIndex + 1, 
                    res,
                    degrees, 
                    ans
                );

                --degrees[u];
                --degrees[v];
            }
        }
    }
\end{lstlisting}
\end{algorithm}


\paragraph{Анализ работы}
Для приведенного алгоритма 